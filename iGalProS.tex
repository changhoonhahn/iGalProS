%\documentclass[iop]{emulateapj}
%\documentclass[12pt, preprint]{emulateapj}
\documentclass[12pt, onecolumn]{emulateapj}

\usepackage{amsmath}
%\usepackage{bibtex}
%\bibliographystyle{unsrtnat}

\usepackage{tikz}
\usetikzlibrary{shapes.geometric, arrows}
\usetikzlibrary{fit}
\usetikzlibrary{arrows.meta}

\tikzstyle{hyper} = [circle, text centered, draw=black]
\tikzstyle{param} = [circle, text centered, draw=black]
\tikzstyle{data} = [circle, text centered, draw=black, line width=2pt]
\tikzstyle{arrow} = [thick,->,>=stealth]
\tikzset{det/.style={insert path={ node {.} }}}

\newcommand{\myemail}{aimalz@nyu.edu}
\newcommand{\textul}{\underline}

%\slugcomment{}

\shorttitle{ABC for SED Fitting}
\shortauthors{[alphabetical authorlist for now?]}

\begin{document}

\begin{align}
\end{align}

\title{iGalProS: a likelihood-free alternative to SED fitting}

\author{ChangHoon Hahn\altaffilmark{1}}
\author{A.I. Malz\altaffilmark{1}}
\author{Nityasri Mandyam\altaffilmark{1}}
\altaffiltext{1}{Center for Cosmology and Particle Physics, Department of Physics, New York University, 4 Washington Pl., room 424, New York, NY 10003, USA}
\email{aimalz@nyu.edu}

\begin{abstract}
It is of interest to obtain galactic star formation rates, star formation histories, stellar masses, and dust compositions, comprising parameters that cannot be directly observed but are rather inferred from properties of galaxy spectra.  The state of the art is spectral energy distribution (SED) fitting, involving [XYZ], which does not involve comparison with the observed galaxy spectra.  This paper proposes an alternative approach based on approximate Bayesian computation (ABC) in which the forward generative model of stellar evolution is evaluated for proposed values of parameters and compared with observations at each step, producing a posterior distribution over the relevant parameters for each galaxy in a survey.
\end{abstract}

\keywords{SED fitting}

\section*{To-do list}
\begin{enumerate}
\end{enumerate}

\section{Introduction}

[Put summaries of papers here]

\acknowledgments{}

\appendix{}

\bibliography{references}

\end{document}