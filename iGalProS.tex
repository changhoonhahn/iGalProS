%\documentclass[iop]{emulateapj}
%\documentclass[12pt, preprint]{emulateapj}
\documentclass[12pt, onecolumn]{emulateapj}

\usepackage{amsmath}
%\usepackage{bibtex}
%\bibliographystyle{unsrtnat}

\usepackage{tikz}
\usetikzlibrary{shapes.geometric, arrows}
\usetikzlibrary{fit}
\usetikzlibrary{arrows.meta}

\tikzstyle{hyper} = [circle, text centered, draw=black]
\tikzstyle{param} = [circle, text centered, draw=black]
\tikzstyle{data} = [circle, text centered, draw=black, line width=2pt]
\tikzstyle{arrow} = [thick,->,>=stealth]
\tikzset{det/.style={insert path={ node {.} }}}

\newcommand{\myemail}{aimalz@nyu.edu}
\newcommand{\textul}{\underline}

%\slugcomment{}

\shorttitle{ABC for SED Fitting}
\shortauthors{[alphabetical authorlist for now?]}

\begin{document}

\begin{align}
\end{align}

\title{iGalProS: a likelihood-free alternative to SED fitting}

\author{ChangHoon Hahn\altaffilmark{1}}
\author{A.I. Malz\altaffilmark{1}}
\author{Nityasri Mandyam\altaffilmark{1}}
\altaffiltext{1}{Center for Cosmology and Particle Physics, Department of Physics, New York University, 4 Washington Pl., room 424, New York, NY 10003, USA}
\email{aimalz@nyu.edu}

\begin{abstract}
It is of interest to obtain galactic star formation rates, star formation histories, stellar masses, and dust compositions, comprising parameters that cannot be directly observed but are rather inferred from properties of galaxy spectra.  High-resolution spectra are too costly to collect for large sample sizes of galaxies, and they may be inaccessible for the dimmest and most distant galaxies, therefore it has become common practice to observe galaxies through several photometric filters and estimate their spectra.  The state of the art is spectral energy distribution (SED) fitting, involving [XYZ], which does not in general involve comparison of the model spectrum with observations.  This paper proposes an alternative approach based on approximate Bayesian computation (ABC) in which the forward generative model of stellar evolution is evaluated for proposed values of parameters and compared with observations at each step, producing an accurate posterior distribution over the relevant parameters for each galaxy in a survey.
\end{abstract}

\keywords{SED fitting}

\section*{To-do list}
\begin{enumerate}
\item
\end{enumerate}

\section{Introduction}

[Put summaries of papers here]

\acknowledgments{}

\appendix{}

\bibliography{references}

\end{document}